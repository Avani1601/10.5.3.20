\documentclass[twocolumn]{article}
\usepackage{amsmath}
\usepackage{booktabs}
\usepackage{tabularx}
\usepackage{caption}

\begin{document}
\title{Discrete Assignment (10.5.3.20)}
\author{Avani Chouhan \\
        EE23BTECH11205}
\date{}  % Remove date
\maketitle

\textbf{Question : }\\
The sum of some terms of G.P. is 315 whose first term and the common ratio are 5 and 2, respectively. Find the last term and the number of terms.\\
\textbf{Solution : }\\
Given:
\begin{align}
x(0) &= 5 \\
r &= 2 \\
x(n) &= x(0)r^n\\
x(z) &= \frac{x(0)}{1-rz^{-1}}\\
S(z) &= X(z)U(z)\\
S(z) &= \frac{x(0)(\frac{r}{1-rz^{-1}}-\frac{1}{1-z^{-1}})}{r-1}
\end{align}
By contour integration:
\begin{equation}
s(n) = x(0)\left(\frac{r^{n+1}-1}{r-1}\right)u(n)
\label{eq:eq1}
\end{equation}

From \eqref{eq:eq1}:
\begin{align}
315 &= 5(2^{n+1}- 1)  \\
63 &= 2^{n+1}-1  \\
64 &= 2^{n+1} \\
n &= 5
\end{align}

\begin{align}
x(n) &= x(0) \cdot r^{n}\\
x(5) &= 5 \cdot 2^{5} \\
 &= 160 
\end{align}

Therefore, the number of terms is 6, and the last term is 160.
\end{document}


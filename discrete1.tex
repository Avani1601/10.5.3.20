\documentclass[twocolumn]{article}
\usepackage{amsmath}
\usepackage{booktabs}
\usepackage{tabularx}
\usepackage{caption}

\begin{document}
\title{Discrete Assignment (10.5.3.20)}
\author{Avani Chouhan \\
        EE23BTECH11205}
\maketitle

\textbf{Question : }\\
The sum of some terms of G.P. is 315 whose first term and the common ratio are 5 and 2, respectively. Find the last term and the number of terms.\\
\textbf{Solution : }\\
Given:
\begin{align}
a &= 5 \label{eq:first_term} \quad &\text{(first term)} \\
r &= 2 \label{eq:common_ratio} \quad &\text{(common ratio)} \\
S_n &= 315 \label{eq:sum_GP} \quad &\text{(sum of the GP)}
\end{align}

Number of terms (\(n\)):
\begin{equation}
315 = \frac{5(2^n - 1)}{2 - 1} \label{eq:num_terms}
\end{equation}

Solving for \(n\):
\begin{align}
315 &= 5(2^n - 1) \notag \\
63 &= 2^n - 1 \notag \\
64 &= 2^n \notag \\
n &= 6 \label{eq:calc_num_terms}
\end{align}

Last term (\(T_n\)):
\begin{align}
T_n &= a \cdot r^{(n-1)} \notag \\
T_n &= 5 \cdot 2^{(6-1)} \label{eq:last_term_calc}
\end{align}

Calculating:
\begin{align}
T_n &= 5 \cdot 32 \notag \\
T_n &= 160 \label{eq:last_term_result}
\end{align}

Therefore, the number of terms is \(n = 6\) (\ref{eq:calc_num_terms}) and the last term is \(T_n = 160\) (\ref{eq:last_term_result}).

Representing the series using Z-transform:
\begin{equation}
X(z) = 5 \sum_{n=0}^{\infty} (2^n)z^{-n} \label{eq:z_transform_series}
\end{equation}

\end{document}


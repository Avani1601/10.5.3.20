\let\negmedspace\undefined
\let\negthickspace\undefined
\documentclass[journal,12pt,twocolumn]{IEEEtran}
\usepackage{cite}
\usepackage{amsmath,amssymb,amsfonts,amsthm}
\usepackage{algorithmic}
\usepackage{graphicx}
\usepackage{textcomp}
\usepackage{xcolor}
\usepackage{txfonts}
\usepackage{listings}
\usepackage{enumitem}
\usepackage{mathtools}
\usepackage{gensymb}
\usepackage{comment}
\usepackage[breaklinks=true]{hyperref}
\usepackage{tkz-euclide}
\usepackage{listings}
\usepackage{gvv}
\def\inputGnumericTable{}
\usepackage[latin1]{inputenc}
\usepackage{color}
\usepackage{array}
\usepackage{longtable}
\usepackage{calc}
\usepackage{multirow}
\usepackage{hhline}
\usepackage{ifthen}
\usepackage{lscape}

\newtheorem{theorem}{Theorem}[section]
\newtheorem{problem}{Problem}
\newtheorem{proposition}{Proposition}[section]
\newtheorem{lemma}{Lemma}[section]
\newtheorem{corollary}[theorem]{Corollary}
\newtheorem{example}{Example}[section]
\newtheorem{definition}[problem]{Definition}
\newcommand{\BEQA}{\begin{eqnarray}}
\newcommand{\EEQA}{\end{eqnarray}}
\newcommand{\define}{\stackrel{\triangle}{=}}
\theoremstyle{remark}
\newtheorem{rem}{Remark}
\begin{document}

\bibliographystyle{IEEEtran}
\vspace{3cm}

\title{NCERT Discrete - 10.5.3.20}
\author{EE23BTECH1205 - Avani Chouhan$^{*}$% <-this % stops a space
}
\maketitle
\newpage
\bigskip

\renewcommand{\thefigure}{\theenumi}
\renewcommand{\thetable}{\theenumi}

\vspace{3cm}
\textbf{Question : 10.5.3.20} 
The sum of some terms of G.P. is 315 whose first term and the common ratio are 5 and 2, respectively. Find the last term and the number of terms.\\
\solution

Given:
\begin{align}
x(0) &= 5 \\
r &= 2 \\
x(n) &= x(0)r^n\\
x(z) &= \frac{x(0)}{1-rz^{-1}}\\
S(z) &= X(z)U(z)\\
S(z) &= \frac{x(0)(\frac{r}{1-rz^{-1}}-\frac{1}{1-z^{-1}})}{r-1}
\end{align}
By contour integration:
\begin{equation}
s(n) = x(0)\left(\frac{r^{n+1}-1}{r-1}\right)u(n)
\label{eq:10.5.3.20eq1}
\end{equation}

From \eqref{eq:10.5.3.20eq1}:
\begin{align}
315 &= 5(2^{n+1}- 1)  \\
63 &= 2^{n+1}-1  \\
64 &= 2^{n+1} \\
n &= 5
\end{align}

\begin{align}
x(n) &= x(0)r^{n}\\
x(5) &= 5\brak{2^{5}}\\
 &= 160 
\end{align}

\begin{figure}
    \centering
    \includegraphics[width=0.8\columnwidth]{figs/graph.png}
    \caption{Stem plot of GP}
    \label{fig:10.5.3.20fig1}
\end{figure}

Therefore, the number of terms is 6, and the last term is 160.
\end{document}

